\documentclass[11 pt]{article}
\title{Proof Portfolio 2}
\usepackage{latexsym}
\usepackage{amssymb}
\usepackage{amsfonts}
\usepackage{amsmath}
\usepackage{amsthm}
\newtheorem{proposition}{Proposition}

\newcommand{\newpar}{\vspace{.15in}\noindent}

\begin{document}

\noindent Jake Irons, Proof 2, Draft 2

\noindent ironsj@mail.gvsu.edu

\newpar
\begin{proposition}
Suppose $abcd$ is a four-digit number. If 11 divides $abc-d$, where $abc$ is the three digit number formed using $a$, $b$, and $c$, then 11 divides $abcd$.
\end{proposition}
\begin{proof}
We will assume that 11 divides $abc-d$, where $abc$ is the three digit number formed using $a$, $b$, $c$, and prove that 11 also divides $abcd$.

\newpar
By the definition of divides, $abc-d=11n$ for some integer $n$, which can also be expressed as $abc=11n+d$. Thus, 
\begin{align*}
abcd&=abc(10)+d \\
&= (11n+d)10+d \\
&= 110n+11d \\
&= 11(11n+d). \\
\end{align*}
\noindent
We will label $k=11n+d$. Because 11, $n$ and $d$ are integers and the set of integers is closed under the operation of multiplication and addition, we know that $k$ is also an integer. Therefore, since
$abcd=11k$ and $k$ is an integer, we can conclude 11 divides $abcd$ by the definition of divides.

\newpar
We have proven that if 11 divides $abc-d$, where $abc$ is the three digit number formed using $a$, $b$, and $c$, then 11 also divides $abcd$.
\end{proof}



\end{document}



