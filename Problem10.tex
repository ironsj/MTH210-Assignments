\documentclass[11 pt]{article}
\title{Proof Portfolio 10}
\usepackage{latexsym}
\usepackage{amssymb}
\usepackage{amsfonts}
\usepackage{amsmath}
\usepackage{amsthm}
\newtheorem{proposition}{Proposition}

\newcommand{\newpar}{\vspace{.15in}\noindent}

\begin{document}

\noindent Jake Irons, Proof 10, Draft 1

\noindent ironsj@mail.gvsu.edu

\newpar
\begin{proposition}
The function $f:\mathbb{N} \to \mathbb{Z}$ defined by

$$ f(n) =  
\begin{cases}      

\frac{n-1}{2} & \textrm{ if $n$ is an odd natural number;} \\   

-\frac{n}{2} & \textrm{ if $n$ is an even natural number.} \\   

\end{cases} $$ 
Then $f$ is a bijection.
\end{proposition}
\begin{proof}
\newpar 
We will prove $f$ is a bijection by first establishing it is an injection and then showing it is a surjection. Let $n_1$ and $n_2$ denote arbitrary elements of the domain $\mathbb{N}$. There are for possible cases based upon whether or not $n_1$ and $n_2$ are even or odd natural numbers:

\textbf{Case 1:} $n_1$ is an odd natural number and $n_2$ is an odd natural number,

\textbf{Case 2:} $n_1$ is an even natural number and $n_2$ is an even natural number,

\textbf{Case 3:} $n_1$ is an even natural number and $n_2$ is an odd natural number,

\textbf{Case 4:} $n_1$ is an odd natural number and $n_2$ is an even natural number.

\newpar
In case 1, since $n_1$ and $n_2$ are both odd, $f(n_1)=\frac{n_1-1}{2}$ and $f(n_2)=\frac{n_2-1}{2}$ by the definition of $f$. Therefore,
\begin{align}
\frac{n_1-1}{2}=\frac{n_2-1}{2}. \label{eq:1} \\
\nonumber \end{align}

\newpar 
By cross multiplying and simplifying (\ref{eq:1}), we get $n_1=n_2$.

\newpar
In case 2, since $n_1$ and $n_2$ are both even, $f(n_1)=-\frac{n_1}{2}$ and $f(n_2)=-\frac{n_2}{2}$ by the definition of $f$. Therefore,
\begin{align}
-\frac{n_1}{2}=-\frac{n_2}{2}. \label{eq:2} \\
\nonumber \end{align}

\newpar By cross multiplying and simplifying (\ref{eq:2}), we get $n_1=n_2$.

\newpar
In case 3, $n_1$ is even and $n_2$ is odd, which implies $n_1\not=n_2$. By the definition of $f$, $f(n_1)\le -1$ and $f(n_2)\ge 0$. Thus, $f(n_1)\not=f(n_2)$.

\newpar
In case 4, $n_1$ is odd and $n_2$ is even, which implies $n_1\not=n_2$. By the definition of $f$, $f(n_1)\ge 0$ and $f(n_2)\le -1$. Thus, $f(n_1)\not=f(n_2)$.


\newpar 
In each of the four cases we have shown, directly or indirectly, that if $n_1\not=n_2$, then $f(n_1)\not=f(n_2)$.

\newpar 
We now prove $f$ is a surjection. Let $y$ denote an arbitrary element of the codomain $\mathbb{Z}$. We will proceed with cases based on the sign of $y$:

\textbf{Case 1:} $y<0$,

\textbf{Case 2:} $y>0$.

\newpar 
In case 1, define $n=-2y$. Since $y<0$, $n$ is an even natural number and thus in the domain $\mathbb{N}$ of $f$. Also, $n$ is even, so that by the definition of $f$,
\begin{align*}
f(n)&=-\frac{(-2y)}{2} \\
&=y. \\
\end{align*}

\newpar 
Thus, $f(n)=y$.

\newpar 
In case 2, define $n=2y+1$. Since $y>0$, $n$ is an odd natural number and thus in the domain $\mathbb{N}$ of $f$. Also, $n$ is odd, so that by the definition of $f$,
\begin{align*}
f(n)&=\frac{(2y+1)-1}{2} \\
&=\frac{2y}{2} \\
&=y. \\
\end{align*}

\newpar 
Thus $f(n)=y$. In each of the two cases we have shown that for an arbitrary element $y$ belonging to the codomain $\mathbb{Z}$ of $f$, there exists $n$ belonging to the domain $\mathbb{N}$ of $f$ such that $f(n)=y$. Thus $f$ is a surjection by the definition of surjection.

\newpar Since $f$ is both an injection and a surjection, it is a bijection by the definition of bijection, which completes the proof.
\end{proof}



\end{document}