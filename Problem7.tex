\documentclass[11 pt]{article}
\title{Proof Portfolio 7}
\usepackage{latexsym}
\usepackage{amssymb}
\usepackage{amsfonts}
\usepackage{amsmath}
\usepackage{amsthm}
\newtheorem{proposition}{Proposition}

\newcommand{\newpar}{\vspace{.15in}\noindent}

\begin{document}

\noindent Jake Irons, Proof 7, Draft 1

\noindent ironsj@mail.gvsu.edu
\newpar
\begin{proposition}
If 5 does not divide the integer $a$, then $a^2\equiv \mbox{1(mod 5)}$ or $a^2\equiv \mbox{1(mod 3)}$.
\end{proposition}
\begin{proof}
We will assume 5 does not divide the integer $a$ and show that $a^2\equiv \mbox{1(mod 5)}$ or $a^2\equiv \mbox{-1(mod 5)}$.

\newpar By the Division Algorithm and the fact 5 does not divide $a$, there are four possible cases to consider based upon integer remainders:

\textbf{Case 1:} $a=5n+1$ where n is an integer, 

\textbf{Case 2:} $a=5n+2$ where n is an integer,

\textbf{Case 3:} $a=5n+3$ where n is an integer, and

\textbf{Case 4:} $a=5n+4$ where n is an integer.

\newpar
\textbf{Case 1:} When $a=5n+1$, we can substitute $a$ into the expression $a^2$ as follows:
\begin{align*}
a^2&=(5n+1)^2 \\
&= 25n^2+10n+1 \\
&= 5(5n^2+2n)+1. \\
\end{align*}
\noindent
We will label $k=5n^2+2n$. Because 5, 2, and $n$ are integers and the set of integers is closed under addition and multiplication, $k$ is an integer. Since $a^2=5k+1$ and $k$ is an integer, $a^2\equiv \mbox{1(mod 5)}$ by the definitions of divides and congruence. 


\newpar
\textbf{Case 2:} When $a=5n+2$, we can substitute $a$ into the expression $a^2$ as follows:
\begin{align*}
a^2&=(5n+2)^2 \\
&= 25n^2+20n+5-1 \\
&= 5(5n^2+4n+1)-1. \\
\end{align*}
\noindent
We will label $k=5n^2+4n+1$. Because 5, 4, 1, and $n$ are integers and the set of integers is closed under addition and multiplication, $k$ is an integer. Since $a^2=5k-1$ and $k$ is an integer, $a^2\equiv \mbox{-1(mod 5)}$ by the definitions of divides and congruence. 


\newpar
\textbf{Case 3:} When $a=5n+3$, we can substitute $a$ into the expression $a^2$ as follows:
\begin{align*}
a^2&=(5n+3)^2 \\
&= 25n^2+30n+10-1 \\
&= 5(5n^2+6n+2)-1. \\
\end{align*}
\noindent
We will label $k=5n^2+6n+2$. Because 5, 6, 2, and $n$ are integers and the set of integers is closed under addition and multiplication, $k$ is an integer. Since $a^2=5k-1$ and $k$ is an integer, $a^2\equiv \mbox{-1(mod 5)}$ by the definitions of divides and congruence. 


\newpar
\textbf{Case 4:} When $a=5n+4$, we can substitute $a$ into the expression $a^2$ as follows:
\begin{align*}
a^2&=(5n+4)^2 \\
&= 25n^2+40n+16 \\
&= 5(5n^2+8n+3)+1. \\
\end{align*}
\noindent
We will label $k=5n^2+8n+3$. Because 5, 8, 3, and $n$ are integers and the set of integers is closed under addition and multiplication, $k$ is an integer. Since $a^2=5k+1$ and $k$ is an integer, $a^2\equiv \mbox{1(mod 5)}$ by the definitions of divides and congruence. 

\newpar
We have proven that the conclusion of the logically equivalent statement is true for all possible cases. Thus, when 5 does not divide the integer $a$, then $a^2\equiv \mbox{1(mod 5)}$ or $a^2\equiv \mbox{-1(mod 5)}$.
\end{proof}



\end{document}



