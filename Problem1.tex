\documentclass[11 pt]{article}
\title{Proof Portfolio 1}
\usepackage{latexsym}
\usepackage{amssymb}
\usepackage{amsfonts}
\usepackage{amsmath}
\usepackage{amsthm}
\newtheorem{proposition}{Proposition}

\newcommand{\newpar}{\vspace{.15in}\noindent}

\begin{document}

\noindent Jake Irons, Proof 1, Draft 1

\noindent ironsj@mail.gvsu.edu

\newpar
\begin{proposition}
If $m$ is an odd integer, then 4 divides $m^3-m^2+m-5$.
\end{proposition}
\begin{proof}
We will assume that $m$ is an odd integer and prove that 4 divides $m^3-m^2+m-5$.

\newpar
By the definition of odd integer, there exists an integer $n$ having the property that $m=2n+1$. Thus,
\begin{align*}
m^3-m^2+m-5&=(2n+1)^3-(2n+1)^2+(2n+1)-5 \\
&=8n^3+8n^2+4n-4 \\
&= 4(2n^3+2n^2+n-1). \\
\end{align*}

\noindent
We will label $k=2n^3+2n^2+n-1$. Because 1,2, and $n$ are integers and the set of integers is closed under the operations of multiplication and addition, we know that $k$ is also an integer. Therefore, since
$m^3-m^2+m-5=4k$ and $k$ is an integer, we can conclude 4 divides $m^3-m^2+m-5$ by the definition of divides. 

\newpar
We have proven that if $m$ is an odd integer, then 4 divides $m^3-m^2+m-5$.
\end{proof}



\end{document}



