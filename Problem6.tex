\documentclass[11 pt]{article}
\title{Proof Portfolio 6}
\usepackage{latexsym}
\usepackage{amssymb}
\usepackage{amsfonts}
\usepackage{amsmath}
\usepackage{amsthm}
\newtheorem{proposition}{Proposition}

\newcommand{\newpar}{\vspace{.15in}\noindent}

\begin{document}

\noindent Jake Irons, Proof 6, Draft 2

\noindent ironsj@mail.gvsu.edu

\newpar
\begin{proposition}
The number $x=\sqrt[]{7}+\sqrt[]{14}$ is irrational.
\end{proposition}
\begin{proof}
We will prove the given statement using proof by contradiction. That is, we will assume that $\sqrt[]{7}+\sqrt[]{14}$ is rational, and show that this assumption leads to a contradiction.

\newpar
By the definition of rational number,
\begin{align}
\sqrt[]{7}+\sqrt[]{14}&=\frac{p}{q}, \label{eq:1} \\
\nonumber \end{align} 
\noindent where $p$ and $q$ are integers and $q\neq0$.

\newpar Squaring both sides of (\ref{eq:1}), we obtain $21+14\sqrt[]{2}=\frac{p^2}{q^2}$, which can be rewritten as 
\begin{align*}
\sqrt[]{2}=\frac{p^2}{14q^2}-\frac{21}{14}. \\
\end{align*}

\newpar
This implies that $\sqrt[]{2}$ is equal to a rational number because 14, 21, and $\frac{p}{q}$ are rational numbers and the set of rational numbers is closed under subtraction and division.  However, by a previously proven proposition, we know $\sqrt[]{2}$ is irrational. Therefore, we have reached a contradiction.

\newpar Having reached a contradiction, we conclude that the number $x=\sqrt[]{7}+\sqrt[]{14}$ is irrational, which completes the proof.
\end{proof}



\end{document}



